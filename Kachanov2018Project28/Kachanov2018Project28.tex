\documentclass[12pt,twoside]{article}
\usepackage{jmlda}
%\NOREVIEWERNOTES
\title
    [Мультимоделирование как универсальный способ описания выборки общего вида] 
    {Мультимоделирование как универсальный способ описания выборки общего вида}
\author
    [Качанов~В.\,В.] % список авторов для колонтитула; не нужен, если основной список влезает в колонтитул
    {Качанов~В.\,В., Соавтор~И.\,О., Фамилия~И.\,О.} % основной список авторов, выводимый в оглавление
\email
    {kachanov.vv@phystech.edu}
\abstract
    {Данный текст является образцом оформления статьи, подаваемой для публикации в журнале <<Машинное обучение и анализ данных>>.
    Аннотация кратко характеризует основную цель работы,
    особенности предлагаемого подхода и~основные результаты.

\bigskip
\textbf{Ключевые слова}: \emph {ключевое слово, ключевое слово,
еще ключевые слова}.}
\begin{document}
\maketitle
%\linenumbers
\section{Введение}
После аннотации, но перед первым разделом,
располагается введение, включающее в себя
описание предметной области,
обоснование актуальности задачи,
краткий обзор известных результатов,
и~т.\,п.


\begin{thebibliography}{1}
\bibitem{voron06latex}
    \BibAuthor{Воронцов~К.\,В.}
    \LaTeXe\ в~примерах.
    2006.
    \BibUrl{http://www.ccas.ru/voron/latex.html}.
\end{thebibliography}

\end{document}
